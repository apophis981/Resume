% Cover letter using letter.cls
\documentclass{letter} % Uses 10pt
%\usepackage{helvetica} % uses helvetica postscript font (download helvetica.sty)
%\usepackage{newcent}   % uses new century schoolbook postscript font 
% the following commands control the margins:
\topmargin=-1in    % Make letterhead start about 1 inch from top of page 
\textheight=8.5in    % text height can be bigger for a longer letter
\oddsidemargin=0pt   % leftmargin is 1 inch
\textwidth=6.5in     % textwidth of 6.5in leaves 1 inch for right margin

\begin{document}

\signature{Jeremy R. Hurst}           % name for signature 
\longindentation=0pt                       % needed to get closing flush left
\let\raggedleft\raggedright                % needed to get date flush left
 
 
\begin{letter}{Linguistic Data Consortium \\
School of Arts and Sciences }



\begin{center}
{\large\bf Jeremy R. Hurst} 
\end{center}
\medskip\hrule height 1pt
\begin{center}
{804 S 48th St. \\ Philadelphia, PA 19143 \\ jeremy.r.hurst@gmail.com \\(661) 231-5273} 
\end{center} \vfill % forces letterhead to top of page
\vspace*{-100mm}% Correct for vertical displacement
\noindent .
\vspace*{50mm}
\opening{To who it may concern:} 
 
\noindent I am applying for the position of Application Developer for the Linguistic Data Consortium at the University of Pennsylvania.

\noindent While I was studying Computer Science at Earlham College, I joined the Theory of Computation research group.
 Our broad focus was on problems in the area of formal language theory, but our specific focus was on computational approaches to studying natural languages.
 Most of the research we did while I was there was on creating finite state atomata that described the formal complexity of stress patterns that occur in the words of human languages.
 I participated in research with the Theory of Computation group for two years until I graduated from Earlham.

\noindent At the point I left, we were working toward simplifying the automata so that they would be easily imported into an application written in haskell.
 The end result would be a computer program which would be able to test if a given string would have a stress pattern order which would fit within the rules of any spoken language.
 I really enjoyed doing linguistics research and collaborating with the linguists and mathematicians in the Theory of Computation group.
 I hope to find a work environment simmilar to that experience and I'm excited about this opportunity. 

\noindent The skills and experience I have acquired working as a Perl Software Developer, the experience I gained as a CS major at Earlham, and my experience doing linguisitcs research with the Theory of Computation group I believe have prepared me so that I meet all of the qualifications listed in the job description.
 I am eager to expand my knowledge in computer science, learn new methodologies, and learn new computer languages.

 \noindent On September 4th I will move to a house in Philadelphia that is very close to the University of Pennsylvania campus.
 I'm hoping to continue my studies in computer science at University of Pennsylvania sometime in the next few years and I hope to get involved with the research being done in the GRASP robotics lab.
 If I were able to secure this position, I would be content to wait until this opportunity is over to begin graduate school. 
 
\closing{Sincerely yours,} 
 

 
\encl{}					% Enclosures

\end{letter}
 

\end{document}
